\documentclass{article}
\usepackage{multirow}
\usepackage[T1]{fontenc}
\usepackage[polish]{babel}
\usepackage[utf8]{inputenc}
\usepackage{hhline}
\usepackage{tabularx}
\usepackage[tableposition=top]{caption}
\usepackage{amsmath}
\usepackage{changepage}
\usepackage{geometry}
\geometry{bottom=10mm, top=10mm}
\usepackage[pdftex,
            pdfauthor={ {{ author }} },
            pdftitle={Projekt na Rachunek prawdopodobieństwa},
            pdfcreationdate={ {{ metadata_date }} },
            pdfmoddate={ {{ metadata_date }} }]{hyperref}
\usepackage[dvipsnames]{xcolor}

\title{Projekt na Rachunek prawdopodobieństwa}
\author{ {{ author }} }
\date{ {{ date }} }

\pagenumbering{gobble}

\begin{document}
\maketitle

\section{Zadanie}
Wyznaczyć równania prostych regresji, według danych zgrupowanych w
tablicy korelacyjnej, powstałej na podstawie mojego numeru indeksu:
\begin{table}[ht!]
    \makebox[\linewidth] {
    \begin{tabular}{|c|c|c|c|c|c|}
        \hline

        {\multirow{2}{*}{$Y$}} &
        \multicolumn{5}{|c|}{$X$}
        \\ \cline{2-6}
          &
        
            
                {{ x_range }}
            
                {{ x_range }} &
            
        
        \\ \cline{1-6}

        
            
                
                    {{ cell }}
                
                    {{ cell}} &
                
            
        \\ \cline{1-6}
        
    \end{tabular}
    }
\end{table}

\section{Rozwiązanie}
W celu wyznaczenia prostych regresji, na podstawie poniższej tabeli, wyznaczamy
najpierw $\overline{x}, \overline{y}, s^2_X$ i $s^2_Y$, a potem ich współczynniki
kierunkowe i wyrazy wolne.
\begin{equation*}
\begin{aligned}[c]
    \overline{x} = \frac{1}{\textcolor{Turquoise}{n}} &\textcolor{YellowOrange}{\sum^{l}_{i=1} \overline{x}_i n_{i.}} = {{ x_mean }} \\
    s^2_X = \frac{1}{\textcolor{Turquoise}{n}} &\textcolor{Purple}{\sum^{l}_{i=1} \overline{x}_i^2 n_{i.}} - {\overline{x}}^2 =
    \frac{1}{ \textcolor{Turquoise}{ {{ n }} } } \cdot \textcolor{Purple}{ {{ x_means_squared_ni }} } - {{ x_mean ** 2 }} =
    {{ x_variance }} \\
\end{aligned}
    \hspace{1em}
\begin{aligned}[c]
    \overline{y} = \frac{1}{\textcolor{Turquoise}{n}} &\textcolor{NavyBlue}{\sum^{m}_{k=1} \overline{y}_k n_{.k}} = {{ y_mean }} \\
    s^2_Y = \frac{1}{\textcolor{Turquoise}{n}} &\textcolor{Green}{\sum^{m}_{k=1} \overline{y}_k^2 n_{.k}} - {\overline{y}}^2 =
    \frac{1}{ \textcolor{Turquoise}{ {{ n }} } } \cdot \textcolor{Green}{ {{ y_means_squared_nk }} } - {{ y_mean ** 2 }} =
    {{ y_variance }} \\
\end{aligned}
\end{equation*}

\begin{table}[ht!]
    \makebox[\linewidth] {
    \begin{tabular}{|c|c|c|c|c|c|c|c|c|c|c|c|}
        \hline

        {\multirow{2}{*}{$\overline{y}_k$}} &
        \multicolumn{5}{|c|}{$\overline{x}_i$ } &
        {\multirow{2}{*}{$n_{.k}$}} &
        {\multirow{2}{*}{$\overline{y}_k n_{.k}$}} &
        {\multirow{2}{*}{$\overline{y}^{2}_{k}$}} &
        {\multirow{2}{*}{$\overline{y}^{2}_{k} n_{.k}$}} &
        {\multirow{2}{*}{$\sum_i \overline{x}_i n_{ik}$}} &
        {\multirow{2}{*}{$\overline{y}_k \sum_i \overline{x}_i n_{ik}$}}
        \\ \cline{2-6}

        \multicolumn{1}{|c|}{} &
        
                {{ x_mean }} &
        
        \\ \cline{1-12}

        
            
                
                    {{ cell }}
                
                    {{ cell }} &
                
            

            
                \\ \cline{1-7} \hhline{============}
            
                \\ \cline{1-12}
            
        

        $n_i$ &
        
            
                \textcolor{Mahogany}{ {{ ni }} }
            
                \textcolor{Turquoise}{ {{ ni }} } &
            
                \textcolor{NavyBlue}{ {{ ni }} } &
            
                \textcolor{Green}{ {{ ni }} } &
            
                {{ ni }} &
            
        
        \\ \cline{1-12}

        $\overline{x}_i n_i$ &
        
            
                \textcolor{YellowOrange}{ {{ x_mean_ni }} }
            
                {{ x_mean_ni }} &
            
        
        \\ \cline{1-7}

        $\overline{x}_{i}^{2}$ &
        
            {{ x_mean_squared }} &
        
        \\ \cline{1-7}


        $\overline{x}_{i}^{2} n_i$ &
        
            
                \textcolor{Purple}{ {{ x_mean_squared_ni }} } 
            
                {{ x_mean_squared_ni }} &
            
        
        \\ \cline{1-7}

    \end{tabular}
    }
\end{table}

\begin{adjustwidth}{-40pt}{0pt}
\begin{equation*}
\begin{aligned}[c]
    a &= \frac{\frac{1}{\textcolor{Turquoise}{n}}\textcolor{Mahogany}{\sum\limits_{k=1}^m \overline{y}_k \sum\limits_{i=1}^l\overline{x}_in_{ik}} - \overline{x}\overline{y}}{s^2_X} =
    \frac{\frac{1}{ \textcolor{Turquoise}{ {{ n }} } } \cdot \textcolor{Mahogany}{ {{ sums_x_means_nik_times_y_mean }} } - {{ x_mean }} \cdot {{ y_mean }} }{\frac{1}{ \textcolor{Turquoise}{ {{ n }} } } \cdot \textcolor{Purple}{ {{ x_means_squared_ni }} }  - {{ x_mean }}^2} = {{ a }}
    & b  = \overline{y} - a\overline{x} = {{ y_mean }} - ({{ a }}) \cdot {{ x_mean }} = {{ b }} & \\
    a' &= \frac{1}{\frac{\frac{1}{\textcolor{Turquoise}{n}}\textcolor{Mahogany}{\sum\limits_{k=1}^m \overline{y}_k \sum\limits_{i=1}^l\overline{x}_in_{ik}} - \overline{x}\overline{y}}{s^2_Y}} =
    \frac{1}{\frac{\frac{1}{ \textcolor{Turquoise}{ {{ n }} } } \cdot \textcolor{Mahogany}{ {{ sums_x_means_nik_times_y_mean }} } - {{ x_mean }} \cdot {{ y_mean }} }{\frac{1}{ \textcolor{Turquoise}{ {{ n }} } } \cdot \textcolor{Green}{ {{ y_means_squared_nk }} } - {{ y_mean }}^2}} = {{ a_prime }}
    & b' = \overline{y} - a'\overline{x} = {{ y_mean }} - ({{ a_prime }})\cdot {{ x_mean }} = {{ b_prime }}
\end{aligned}
\end{equation*}
\end{adjustwidth}

\begin{adjustwidth}{-40pt}{0pt}
Zatem, równanie prostej regresji cechy Y względem cechy X to: $y = {{ a }}x + {{ b }}$ \\
Zatem, równanie prostej regresji cechy X względem cechy Y to: $y = {{ a_prime }}x + {{ b_prime }}$
\end{adjustwidth}
\end{document}
