\documentclass{article}
\usepackage{multirow}
\usepackage[T1]{fontenc}
\usepackage[polish]{babel}
\usepackage[utf8]{inputenc}
\usepackage{hhline}
\usepackage{tabularx}
\usepackage[tableposition=top]{caption}

\begin{document}
\section{Tablica korelacyjna dla mojego numeru indeksu}
\begin{table}[ht!]
    \makebox[\linewidth] {
    \begin{tabular}{|c|c|c|c|c|c|}
        \hline

        {\multirow{2}{*}{$Y$}} &
        \multicolumn{5}{|c|}{$X$}
        \\ \cline{2-6}
          &
        
            
                {{ x_range }}
            
                {{ x_range }} &
            
        
        \\ \cline{1-6}

        
            
                
                    {{ cell }}
                
                    {{ cell}} &
                
            
        \\ \cline{1-6}
        
    \end{tabular}
    }
\end{table}

\section{Tabela z obliczeniami pomocniczymi}
\begin{table}[ht!]
    \makebox[\linewidth] {
    \begin{tabular}{|c|c|c|c|c|c|c|c|c|c|c|c|}
        \hline

        {\multirow{2}{*}{$\overline{y}_k$}} &
        \multicolumn{5}{|c|}{$\overline{x}_i$ } &
        {\multirow{2}{*}{$n_{.k}$}} &
        {\multirow{2}{*}{$\overline{y}_k n_{.k}$}} &
        {\multirow{2}{*}{$\overline{y}^{2}_{k}$}} &
        {\multirow{2}{*}{$\overline{y}^{2}_{k} n_{.k}$}} &
        {\multirow{2}{*}{$\sum_i \overline{x}_i n_{ik}$}} &
        {\multirow{2}{*}{$\overline{y}_k \sum_i \overline{x}_i n_{ik}$}}
        \\ \cline{2-6}

        \multicolumn{1}{|c|}{} &
        
                {{ x_mean }} &
        
        \\ \cline{1-12}

        
            
                
                    {{ cell }}
                
                    {{ cell }} &
                
            

            
                \\ \cline{1-7} \hhline{============}
            
                \\ \cline{1-12}
            
        

        $n_i$ &
        
            
                {{ ni }}
            
                {{ ni }} &
            
        
        \\ \cline{1-12}

        $\overline{x}_i n_i$ &
        
            
                {{ x_mean_ni }}
            
                {{ x_mean_ni }} &
            
        
        \\ \cline{1-7}

        $\overline{x}_{i}^{2}$ &
        
            {{ x_mean_squared }} &
        
        \\ \cline{1-7}


        $\overline{x}_{i}^{2} n_i$ &
        
            
                {{ x_mean_squared_ni }}
            
                {{ x_mean_squared_ni }} &
            
        
        \\ \cline{1-7}

    \end{tabular}
    }
\end{table}

\section{Równania prostych}
Równanie prostej regresji cechy Y względem cechy X, na podstawie próbki ma postać:
        \[y = {{ a }}x + {{ b }}\]

Równanie prostej regresji cechy X względem cechy Y, na podstawie próbki ma postać:
        \[y = {{ a_prime }}x + {{ b_prime }}\]
\end{document}
